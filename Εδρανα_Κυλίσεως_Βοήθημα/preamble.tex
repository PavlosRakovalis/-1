


%URL: Μπορείς να βάλεις url με την εντολή \url{} και μετά καθορίζει οτι τα γράμματα του url θα είναι μπλέ
\usepackage[hidelinks,draft=false]{hyperref}
\hypersetup{colorlinks,linkcolor={blue!40!black!95!green},citecolor={blue!50!black},urlcolor={cyan!70!black}}

%%% ΑΠΟ ΕΔΩ ΚΑΙ ΠΕΡΑ ΕΙΝΑΙ ΟΙ ΡΥΘΜΙΣΕΙΣ ΤΗΣ ASAT

\usepackage[utf8]{inputenc}
%allow simultaneous greek and english input
\usepackage[greek,english]{babel}
\usepackage{alphabeta}
%package needed for ASAT image template
\usepackage{eso-pic}
%colour links, hrefs, etc whatever colour you like
%use only if needed. Shows warnings on Greek characters but nothing bad actually happens.
%just annoys you


%Τα παρακάτω μπήκαν σε σχολιο καθώς φαινεται οτι δημιοργούσαν καποιο πρόβλημα
%\usepackage[colorlinks = true,
%            linkcolor = black,
%            urlcolor  = blue,
%            citecolor = blue,
%            anchorcolor = blue,
%            unicode]{hyperref}

%create a macro named BackgroundPic for setting a background on the first page
\newcommand\BackgroundPic{
    \put(-3.4,0){
    \parbox[b][\paperheight]{\paperwidth}{%
    \vfill
    \centering
    \includegraphics[width=\paperwidth,height=\paperheight]{AERONAUTICS_REPORT_TEMPLATE.pdf}
    \vfill
    }}}
%Info
%turn contents name to Greek
\addto\captionsenglish{
    \renewcommand{\contentsname}{Contents}}


    %include page geometry packages
\usepackage{natbib}
\usepackage{graphicx}
\usepackage[a4paper, total={6in, 8in}]{geometry}


%Προσθέτω την ικανότητα να μπορώ να δημιουργώ κουτάκια hint με χρήση της εντολής \begin{hint}{πχ συνάρτηση μεταφοράς}{} \end{hint}
\usepackage[skins,theorems]{tcolorbox}
\newtcbtheorem[number within=section,list inside=hint]{hint}{hint}%
{colback=blue!5,colframe=cyan!35!black,colbacktitle=blue!35!black,fonttitle=\bfseries,enhanced,attach boxed title to top left={yshift=-2mm,xshift=-2mm}}{def}

%Δημιουργώ την εντολή \red{} η οποία μου επιτρέπει να γράφω κόκκινο κείμενο 
\newcommand{\red}[1]{\textcolor{red}{#1}}
